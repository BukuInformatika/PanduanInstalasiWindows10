\section{Instalasi Windows 10 Dengan Flashdisk}
Tutorial kali ini mengenai cara menginstall windows 10 dengan flashdisk. Kebanyakan orang lebih suka menggunakan flashdisk sebagai media instalasi daripada menggunakan DVD. Alasan pertama mungkin karena notebook yang digunakan tidak dilengkapi dengan CD drive atau DVD drive PC rusak. Alasan kedua karena flashdisk lebih simpel. 

Untuk melakukan instalasi windows 10 menggunakan flashdisk, terlebih dahulu kita harus menjadikan flashdisk menjadi bootable. Ada banyak jenis aplikasi atau program yang bisa digunakan untuk menjadikan flashdisk menjadi bootable. Dalam tutorial ini, saya merekomendasikan menggunakan aplikasi Rufus. 

Sebenarnya tanpa menggunakan program pihak ketiga pun kita bisa membootable flashdisk yaitu dengan menggunakan tool CMD dari windows. Tapi tentu saja, untuk orang awam di dunia komputer sangat tidak disarankan untuk menggunakannya. 

\subsection{Siapkan Flasdisk Terlebih Dahulu}
Sebelum itu, pastikan anda sudah menyiapkan fashdisk berukuran minimal 4GB atau 8GB. Dan yang kedua adalah siapkan file ISO windowsnya. 


